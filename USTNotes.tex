\documentclass[12pt]{article}
\usepackage[a4paper,margin=1in]{geometry}
\usepackage{graphicx}
\usepackage{fancyhdr}
\usepackage{setspace}
\usepackage{titlesec}
\usepackage{hyperref}
\usepackage{background}
\usepackage{listings}

\newcommand{\ClassName}{Operating System}
\newcommand{\Professor}{Dawen CHEN}
\newcommand{\ClassNum}{Comp 1111}
\newcommand{\Author}{Author}

% --- Custom Colors ---
\definecolor{codebg}{rgb}{0.97,0.97,0.97}
\definecolor{codeframe}{rgb}{0.7,0.7,0.7}

% Header/Footer
\pagestyle{fancy}
\fancyhf{}
\fancyhead[L]{Lecture Notes for}
\fancyhead[R]{\textbf{\ClassName}}
\fancyfoot[C]{\thepage}

\backgroundsetup{
	scale=0.75,
	angle=0,
	position=current page.center,
	opacity=0.15, % for transparency!
	contents={\includegraphics[height=\paperheight,width=\paperwidth]{background.png}}
}

% Section formatting
\titleformat{\section}{\large\bfseries}{\thesection.}{1em}{}
\titleformat{\subsection}{\normalsize\bfseries}{\thesubsection}{1em}{}

%% ---- LESSON ENVIRONMENT ----
%\newcounter{lesson}
%\renewcommand{\thelesson}{\arabic{lesson}}
%\newcounter{point}[lesson]
%\renewcommand{\thepoint}{\arabic{point}}
%\newenvironment{lesson}[1]{%
%    \refstepcounter{lesson}%
%    \setcounter{point}{0}%
%    \clearpage
%    \section*{Lesson \thelesson: #1}
%    \addcontentsline{toc}{section}{Lesson \thelesson: #1}
%}{%
%    \par\vspace{2em}\hrule
%}
%\newenvironment{point}[1]{%
%    \refstepcounter{point}%
%    \par\noindent\textbf{Point~\thepoint.~#1}\par
%}{\par\vspace{0.7em}}

% Listings setup for code display
\lstset{
  backgroundcolor=\color{codebg},
  frame=single,
  rulecolor=\color{codeframe},
  basicstyle=\ttfamily\medium,
  keywordstyle=\color{blue!70},
  commentstyle=\color{red!50!green!50!blue!50},
  frame=shadowbox,
  rulesepcolor=\color{red!20!green!20!blue!20},
  breaklines=true,
  columns=fullflexible,
  keepspaces=true,
  captionpos=b,
  aboveskip=1em,
  belowskip=1em,
  xleftmargin=1em,
  xrightmargin=1em,
  framerule=0.7pt,
  showstringspaces=false,
  numbers=left,
  numberstyle=\tiny,
}


\begin{document}

% ---------------- COVER PAGE ----------------
\begin{titlepage}
	\NoBgThispage
    \centering
    % School Logo Placeholder
	\begin{figure}
		\centering
		\includegraphics[width=5cm]{UST.png}
	\end{figure}
    \vspace{1cm}
    {\Huge \bfseries \ClassName \par}
    \vspace{1cm}
    {\Large Lecture Notes \par}
    \vspace{1cm}
    {\large Professor: \textbf{\Professor} \par}
    \vspace{0.5cm}
    {\large Author: \textbf{\Author} \par}
    \vfill
    {\large From \today \par}
    \vspace{0.5cm}
    {\large To \today \par}
\end{titlepage}

% ---------------- TABLE OF CONTENTS ----------------
\tableofcontents
\newpage
% ---------------- CONTENT ----------------

\begin{section}{Basic Usage}
\begin{subsection}{Subsection}
\textbf{You can use normal \LaTeX\ grammar inside this scope.}

\begin{subsection}{For example}
You can make an unordered list, like this:
\begin{itemize}
	\item This is the first item
	\item This is the second item
\end{itemize}

Or you can make an ordered list, like this:
\begin{enumerate}
	\item This is the first item
	\item This is the second item
\end{enumerate}

You can also insert a math formula both inline: $E = mc^2$ and centering $$E = mc^2$$


\begin{subsection}{Create a code block}
\begin{lstlisting}[language=C, caption=Example C code]
	#include <stdio.h>
	
	int main() {
		printf("Hello HKUST\n"); // this is a comment
		return 0;
	}
\end{lstlisting}

\begin{lstlisting}[language=Python, caption=Example Python code]
	print("Hello HKUST")
\end{lstlisting}

% Add more sections/subsections as needed

\end{document}